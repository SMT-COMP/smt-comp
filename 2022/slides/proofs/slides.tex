%%%%%%%%%%%%%%%%%%%%%%%%%%%%%%%%%%%%%%%%%%%%%%%%%%%%%%%%%%%%%%%%%%%%%%%%
%
%   SMT-COMP 2022 - Presentation on SMT-Workshop
%                   virtual conference in "Los Angeles"
%
%   1 hour
%
%%%%%%%%%%%%%%%%%%%%%%%%%%%%%%%%%%%%%%%%%%%%%%%%%%%%%%%%%%%%%%%%%%%%%%%%

\documentclass[table]{beamer}
\usepackage[utf8]{inputenc}
\usepackage{xcolor}
\usepackage{tikz}
\usetikzlibrary{shapes,shapes.callouts,automata,trees}
\usetikzlibrary{decorations.pathmorphing,external,fit}
\usetikzlibrary{calc}
\usetikzlibrary{backgrounds} %used for the CEGAR figure
\usepackage{amssymb}
\usepackage{clrscode}
\usepackage{pifont}
\usepackage{pdfpages}
\geometry{papersize={16cm,9cm}}
%\tikzexternalize

\colorlet{MYred}{red!70!black}
\definecolor{MYgreen}{rgb}{.1,.5,0}
\definecolor{MYblue}{rgb}{0,.42,.714}
\colorlet{MYgray}{white!95!MYblue}
\colorlet{MYorange}{orange!80!black}
\definecolor{gold}{rgb}{.8,.6,0}
\colorlet{silver}{white!55!black}
\colorlet{bronze}{brown!70!black}
\def\tick{\ding{52}}
\def\cross{\ding{54}}

%%%%%%%%%%%%%%%%%%%%
%%% Beamer stuff %%%
%%%%%%%%%%%%%%%%%%%%
\usetheme{default}
\useinnertheme{rounded}
\setbeamertemplate{frametitle}[default][center]
\setbeamertemplate{footline}{\quad\hfill\footnotesize\insertframenumber\strut\kern1em\vskip2pt}
\setbeamertemplate{navigation symbols}{}
\setbeamertemplate{itemize/enumerate subbody begin}{\normalsize}
\usefonttheme[onlymath]{serif} % Nicer formulas
\setbeamercolor{block body}{bg=black!10}
\setbeamercolor{block title}{bg=black!20}

\AtBeginSection[]{
  \begin{frame}
  \vfill
  \centering
  \begin{beamercolorbox}[sep=8pt,center,shadow=true,rounded=true]{title}
    \usebeamerfont{title}\insertsectionhead\par%
  \end{beamercolorbox}
  \vfill
  \end{frame}
}

\def\emph#1{\textcolor{MYblue}{#1}}

%%% Titel, Autor und Datum des Vortrags:
\title{SMT-COMP 2022\\
17th International Satisfiability Modulo Theory Competition}
\author{\emph{Haniel Barbosa} \and Jochen Hoenicke \and Fran\c{c}ois Bobot}
\date{Aug 11, 2022}

%% Institut
\institute{
  Universidade Federal de Minas Gerais, Brazil \and
  Albert-Ludwigs-Universit\"at Freiburg, Germany \and
  CEA List, France
}


%%%%%%%%%%%%%%%%%%%%%%%%%%%%%%%
% MACROS

% database symbol (from stackexchange)

\makeatletter
\tikzset{
    database/.style={
        path picture={
            \draw (0, 1.5*\database@segmentheight) circle [x radius=\database@radius,y radius=\database@aspectratio*\database@radius];
            \draw (-\database@radius, 0.5*\database@segmentheight) arc [start angle=180,end angle=360,x radius=\database@radius, y radius=\database@aspectratio*\database@radius];
            \draw (-\database@radius,-0.5*\database@segmentheight) arc [start angle=180,end angle=360,x radius=\database@radius, y radius=\database@aspectratio*\database@radius];
            \draw (-\database@radius,1.5*\database@segmentheight) -- ++(0,-3*\database@segmentheight) arc [start angle=180,end angle=360,x radius=\database@radius, y radius=\database@aspectratio*\database@radius] -- ++(0,3*\database@segmentheight);
        },
        minimum width=2*\database@radius + \pgflinewidth,
        minimum height=3*\database@segmentheight + 2*\database@aspectratio*\database@radius + \pgflinewidth,
    },
    database segment height/.store in=\database@segmentheight,
    database radius/.store in=\database@radius,
    database aspect ratio/.store in=\database@aspectratio,
    database segment height=0.1cm,
    database radius=0.25cm,
    database aspect ratio=0.35,
}
\makeatother


%%%%%%%%%%%%%%%%%%%%%%%%%%%%%%%

\newcommand\vitem{\vfill\item}

\begin{document}


\begin{frame}
  \frametitle{SMT Proofs}

  \begin{itemize}
    \item In first meeting we discussed:
    \begin{itemize}
      \item Detailed vs coarse proofs (for scalability)
      \begin{itemize}
        \item Similarly, to dump proofs as solving goes rather than store and
        emit in the end
      \end{itemize}

      \item Logical framework or predefined proof calculus

      \item Proof structure and BNF for a proof format

    \end{itemize}
    \item In the second meeting we mostly discussed the last point
    \begin{itemize}
      \item Tentative consensus towards using SMT-LIB 3.0
      \item Same language for proofs and (regular) terms? Could help to separate syntactically.
      \item Expectation of proof as sequence of steps towards proving something
      \begin{itemize}
        \item $(c_1 \dots c_n\ p)$ as the response to get proof, with $c_i$ as
        preamble, $p$ the proof
        \item A proof could be $(\mathrm{someRule} :\mathrm{conclusion}\ t :\mathrm{premises}\ (\mathrm{list}\
        p_1 \dots p_n\ p) :\mathrm{args}\ (\mathrm{list}\ t_1\dots t_n))$
        \item Proof can be flattened (so not just a big proof term) by naming
        proofs in the preamble
      \end{itemize}
    \end{itemize}
  \end{itemize}
\end{frame}

\begin{frame}[fragile,t,fragile]
  \frametitle{SMT Proofs}

  Two suggestions by Andrew Reynolds for defining a rule via SMT-LIB 3.0. One
  with simple types:

\begin{verbatim}
(declare-type Proof ())
...
(declare-const proof.res (-> Proof Proof Bool Proof))
\end{verbatim}

Another with more structured types:

\begin{verbatim}
(declare-type Proof (Bool))
...
(declare-const proof.eq_symm
  (-> (! Type :var T :implicit)
      (! T :var x :implicit)
      (! T :var y :implicit)
      (Proof (= x y))
      (Proof (= y x))))
\end{verbatim}

    \medskip

    so that for example the proof \verb|(proof.eq_symm P)| proves \verb|(= y x)| when \verb|P| proves \verb|(= x y)|.




\end{frame}


\end{document}
